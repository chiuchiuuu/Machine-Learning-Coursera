
% Default to the notebook output style

    


% Inherit from the specified cell style.




    
\documentclass[11pt]{article}

    
    
    \usepackage[T1]{fontenc}
    % Nicer default font (+ math font) than Computer Modern for most use cases
    \usepackage{mathpazo}

    % Basic figure setup, for now with no caption control since it's done
    % automatically by Pandoc (which extracts ![](path) syntax from Markdown).
    \usepackage{graphicx}
    % We will generate all images so they have a width \maxwidth. This means
    % that they will get their normal width if they fit onto the page, but
    % are scaled down if they would overflow the margins.
    \makeatletter
    \def\maxwidth{\ifdim\Gin@nat@width>\linewidth\linewidth
    \else\Gin@nat@width\fi}
    \makeatother
    \let\Oldincludegraphics\includegraphics
    % Set max figure width to be 80% of text width, for now hardcoded.
    \renewcommand{\includegraphics}[1]{\Oldincludegraphics[width=.8\maxwidth]{#1}}
    % Ensure that by default, figures have no caption (until we provide a
    % proper Figure object with a Caption API and a way to capture that
    % in the conversion process - todo).
    \usepackage{caption}
    \DeclareCaptionLabelFormat{nolabel}{}
    \captionsetup{labelformat=nolabel}

    \usepackage{adjustbox} % Used to constrain images to a maximum size 
    \usepackage{xcolor} % Allow colors to be defined
    \usepackage{enumerate} % Needed for markdown enumerations to work
    \usepackage{geometry} % Used to adjust the document margins
    \usepackage{amsmath} % Equations
    \usepackage{amssymb} % Equations
    \usepackage{textcomp} % defines textquotesingle
    % Hack from http://tex.stackexchange.com/a/47451/13684:
    \AtBeginDocument{%
        \def\PYZsq{\textquotesingle}% Upright quotes in Pygmentized code
    }
    \usepackage{upquote} % Upright quotes for verbatim code
    \usepackage{eurosym} % defines \euro
    \usepackage[mathletters]{ucs} % Extended unicode (utf-8) support
    \usepackage[utf8x]{inputenc} % Allow utf-8 characters in the tex document
    \usepackage{fancyvrb} % verbatim replacement that allows latex
    \usepackage{grffile} % extends the file name processing of package graphics 
                         % to support a larger range 
    % The hyperref package gives us a pdf with properly built
    % internal navigation ('pdf bookmarks' for the table of contents,
    % internal cross-reference links, web links for URLs, etc.)
    \usepackage{hyperref}
    \usepackage{longtable} % longtable support required by pandoc >1.10
    \usepackage{booktabs}  % table support for pandoc > 1.12.2
    \usepackage[inline]{enumitem} % IRkernel/repr support (it uses the enumerate* environment)
    \usepackage[normalem]{ulem} % ulem is needed to support strikethroughs (\sout)
                                % normalem makes italics be italics, not underlines
    

    
    
    % Colors for the hyperref package
    \definecolor{urlcolor}{rgb}{0,.145,.698}
    \definecolor{linkcolor}{rgb}{.71,0.21,0.01}
    \definecolor{citecolor}{rgb}{.12,.54,.11}

    % ANSI colors
    \definecolor{ansi-black}{HTML}{3E424D}
    \definecolor{ansi-black-intense}{HTML}{282C36}
    \definecolor{ansi-red}{HTML}{E75C58}
    \definecolor{ansi-red-intense}{HTML}{B22B31}
    \definecolor{ansi-green}{HTML}{00A250}
    \definecolor{ansi-green-intense}{HTML}{007427}
    \definecolor{ansi-yellow}{HTML}{DDB62B}
    \definecolor{ansi-yellow-intense}{HTML}{B27D12}
    \definecolor{ansi-blue}{HTML}{208FFB}
    \definecolor{ansi-blue-intense}{HTML}{0065CA}
    \definecolor{ansi-magenta}{HTML}{D160C4}
    \definecolor{ansi-magenta-intense}{HTML}{A03196}
    \definecolor{ansi-cyan}{HTML}{60C6C8}
    \definecolor{ansi-cyan-intense}{HTML}{258F8F}
    \definecolor{ansi-white}{HTML}{C5C1B4}
    \definecolor{ansi-white-intense}{HTML}{A1A6B2}

    % commands and environments needed by pandoc snippets
    % extracted from the output of `pandoc -s`
    \providecommand{\tightlist}{%
      \setlength{\itemsep}{0pt}\setlength{\parskip}{0pt}}
    \DefineVerbatimEnvironment{Highlighting}{Verbatim}{commandchars=\\\{\}}
    % Add ',fontsize=\small' for more characters per line
    \newenvironment{Shaded}{}{}
    \newcommand{\KeywordTok}[1]{\textcolor[rgb]{0.00,0.44,0.13}{\textbf{{#1}}}}
    \newcommand{\DataTypeTok}[1]{\textcolor[rgb]{0.56,0.13,0.00}{{#1}}}
    \newcommand{\DecValTok}[1]{\textcolor[rgb]{0.25,0.63,0.44}{{#1}}}
    \newcommand{\BaseNTok}[1]{\textcolor[rgb]{0.25,0.63,0.44}{{#1}}}
    \newcommand{\FloatTok}[1]{\textcolor[rgb]{0.25,0.63,0.44}{{#1}}}
    \newcommand{\CharTok}[1]{\textcolor[rgb]{0.25,0.44,0.63}{{#1}}}
    \newcommand{\StringTok}[1]{\textcolor[rgb]{0.25,0.44,0.63}{{#1}}}
    \newcommand{\CommentTok}[1]{\textcolor[rgb]{0.38,0.63,0.69}{\textit{{#1}}}}
    \newcommand{\OtherTok}[1]{\textcolor[rgb]{0.00,0.44,0.13}{{#1}}}
    \newcommand{\AlertTok}[1]{\textcolor[rgb]{1.00,0.00,0.00}{\textbf{{#1}}}}
    \newcommand{\FunctionTok}[1]{\textcolor[rgb]{0.02,0.16,0.49}{{#1}}}
    \newcommand{\RegionMarkerTok}[1]{{#1}}
    \newcommand{\ErrorTok}[1]{\textcolor[rgb]{1.00,0.00,0.00}{\textbf{{#1}}}}
    \newcommand{\NormalTok}[1]{{#1}}
    
    % Additional commands for more recent versions of Pandoc
    \newcommand{\ConstantTok}[1]{\textcolor[rgb]{0.53,0.00,0.00}{{#1}}}
    \newcommand{\SpecialCharTok}[1]{\textcolor[rgb]{0.25,0.44,0.63}{{#1}}}
    \newcommand{\VerbatimStringTok}[1]{\textcolor[rgb]{0.25,0.44,0.63}{{#1}}}
    \newcommand{\SpecialStringTok}[1]{\textcolor[rgb]{0.73,0.40,0.53}{{#1}}}
    \newcommand{\ImportTok}[1]{{#1}}
    \newcommand{\DocumentationTok}[1]{\textcolor[rgb]{0.73,0.13,0.13}{\textit{{#1}}}}
    \newcommand{\AnnotationTok}[1]{\textcolor[rgb]{0.38,0.63,0.69}{\textbf{\textit{{#1}}}}}
    \newcommand{\CommentVarTok}[1]{\textcolor[rgb]{0.38,0.63,0.69}{\textbf{\textit{{#1}}}}}
    \newcommand{\VariableTok}[1]{\textcolor[rgb]{0.10,0.09,0.49}{{#1}}}
    \newcommand{\ControlFlowTok}[1]{\textcolor[rgb]{0.00,0.44,0.13}{\textbf{{#1}}}}
    \newcommand{\OperatorTok}[1]{\textcolor[rgb]{0.40,0.40,0.40}{{#1}}}
    \newcommand{\BuiltInTok}[1]{{#1}}
    \newcommand{\ExtensionTok}[1]{{#1}}
    \newcommand{\PreprocessorTok}[1]{\textcolor[rgb]{0.74,0.48,0.00}{{#1}}}
    \newcommand{\AttributeTok}[1]{\textcolor[rgb]{0.49,0.56,0.16}{{#1}}}
    \newcommand{\InformationTok}[1]{\textcolor[rgb]{0.38,0.63,0.69}{\textbf{\textit{{#1}}}}}
    \newcommand{\WarningTok}[1]{\textcolor[rgb]{0.38,0.63,0.69}{\textbf{\textit{{#1}}}}}
    
    
    % Define a nice break command that doesn't care if a line doesn't already
    % exist.
    \def\br{\hspace*{\fill} \\* }
    % Math Jax compatability definitions
    \def\gt{>}
    \def\lt{<}
    % Document parameters
    \title{Untitled}
    
    
    

    % Pygments definitions
    
\makeatletter
\def\PY@reset{\let\PY@it=\relax \let\PY@bf=\relax%
    \let\PY@ul=\relax \let\PY@tc=\relax%
    \let\PY@bc=\relax \let\PY@ff=\relax}
\def\PY@tok#1{\csname PY@tok@#1\endcsname}
\def\PY@toks#1+{\ifx\relax#1\empty\else%
    \PY@tok{#1}\expandafter\PY@toks\fi}
\def\PY@do#1{\PY@bc{\PY@tc{\PY@ul{%
    \PY@it{\PY@bf{\PY@ff{#1}}}}}}}
\def\PY#1#2{\PY@reset\PY@toks#1+\relax+\PY@do{#2}}

\expandafter\def\csname PY@tok@w\endcsname{\def\PY@tc##1{\textcolor[rgb]{0.73,0.73,0.73}{##1}}}
\expandafter\def\csname PY@tok@c\endcsname{\let\PY@it=\textit\def\PY@tc##1{\textcolor[rgb]{0.25,0.50,0.50}{##1}}}
\expandafter\def\csname PY@tok@cp\endcsname{\def\PY@tc##1{\textcolor[rgb]{0.74,0.48,0.00}{##1}}}
\expandafter\def\csname PY@tok@k\endcsname{\let\PY@bf=\textbf\def\PY@tc##1{\textcolor[rgb]{0.00,0.50,0.00}{##1}}}
\expandafter\def\csname PY@tok@kp\endcsname{\def\PY@tc##1{\textcolor[rgb]{0.00,0.50,0.00}{##1}}}
\expandafter\def\csname PY@tok@kt\endcsname{\def\PY@tc##1{\textcolor[rgb]{0.69,0.00,0.25}{##1}}}
\expandafter\def\csname PY@tok@o\endcsname{\def\PY@tc##1{\textcolor[rgb]{0.40,0.40,0.40}{##1}}}
\expandafter\def\csname PY@tok@ow\endcsname{\let\PY@bf=\textbf\def\PY@tc##1{\textcolor[rgb]{0.67,0.13,1.00}{##1}}}
\expandafter\def\csname PY@tok@nb\endcsname{\def\PY@tc##1{\textcolor[rgb]{0.00,0.50,0.00}{##1}}}
\expandafter\def\csname PY@tok@nf\endcsname{\def\PY@tc##1{\textcolor[rgb]{0.00,0.00,1.00}{##1}}}
\expandafter\def\csname PY@tok@nc\endcsname{\let\PY@bf=\textbf\def\PY@tc##1{\textcolor[rgb]{0.00,0.00,1.00}{##1}}}
\expandafter\def\csname PY@tok@nn\endcsname{\let\PY@bf=\textbf\def\PY@tc##1{\textcolor[rgb]{0.00,0.00,1.00}{##1}}}
\expandafter\def\csname PY@tok@ne\endcsname{\let\PY@bf=\textbf\def\PY@tc##1{\textcolor[rgb]{0.82,0.25,0.23}{##1}}}
\expandafter\def\csname PY@tok@nv\endcsname{\def\PY@tc##1{\textcolor[rgb]{0.10,0.09,0.49}{##1}}}
\expandafter\def\csname PY@tok@no\endcsname{\def\PY@tc##1{\textcolor[rgb]{0.53,0.00,0.00}{##1}}}
\expandafter\def\csname PY@tok@nl\endcsname{\def\PY@tc##1{\textcolor[rgb]{0.63,0.63,0.00}{##1}}}
\expandafter\def\csname PY@tok@ni\endcsname{\let\PY@bf=\textbf\def\PY@tc##1{\textcolor[rgb]{0.60,0.60,0.60}{##1}}}
\expandafter\def\csname PY@tok@na\endcsname{\def\PY@tc##1{\textcolor[rgb]{0.49,0.56,0.16}{##1}}}
\expandafter\def\csname PY@tok@nt\endcsname{\let\PY@bf=\textbf\def\PY@tc##1{\textcolor[rgb]{0.00,0.50,0.00}{##1}}}
\expandafter\def\csname PY@tok@nd\endcsname{\def\PY@tc##1{\textcolor[rgb]{0.67,0.13,1.00}{##1}}}
\expandafter\def\csname PY@tok@s\endcsname{\def\PY@tc##1{\textcolor[rgb]{0.73,0.13,0.13}{##1}}}
\expandafter\def\csname PY@tok@sd\endcsname{\let\PY@it=\textit\def\PY@tc##1{\textcolor[rgb]{0.73,0.13,0.13}{##1}}}
\expandafter\def\csname PY@tok@si\endcsname{\let\PY@bf=\textbf\def\PY@tc##1{\textcolor[rgb]{0.73,0.40,0.53}{##1}}}
\expandafter\def\csname PY@tok@se\endcsname{\let\PY@bf=\textbf\def\PY@tc##1{\textcolor[rgb]{0.73,0.40,0.13}{##1}}}
\expandafter\def\csname PY@tok@sr\endcsname{\def\PY@tc##1{\textcolor[rgb]{0.73,0.40,0.53}{##1}}}
\expandafter\def\csname PY@tok@ss\endcsname{\def\PY@tc##1{\textcolor[rgb]{0.10,0.09,0.49}{##1}}}
\expandafter\def\csname PY@tok@sx\endcsname{\def\PY@tc##1{\textcolor[rgb]{0.00,0.50,0.00}{##1}}}
\expandafter\def\csname PY@tok@m\endcsname{\def\PY@tc##1{\textcolor[rgb]{0.40,0.40,0.40}{##1}}}
\expandafter\def\csname PY@tok@gh\endcsname{\let\PY@bf=\textbf\def\PY@tc##1{\textcolor[rgb]{0.00,0.00,0.50}{##1}}}
\expandafter\def\csname PY@tok@gu\endcsname{\let\PY@bf=\textbf\def\PY@tc##1{\textcolor[rgb]{0.50,0.00,0.50}{##1}}}
\expandafter\def\csname PY@tok@gd\endcsname{\def\PY@tc##1{\textcolor[rgb]{0.63,0.00,0.00}{##1}}}
\expandafter\def\csname PY@tok@gi\endcsname{\def\PY@tc##1{\textcolor[rgb]{0.00,0.63,0.00}{##1}}}
\expandafter\def\csname PY@tok@gr\endcsname{\def\PY@tc##1{\textcolor[rgb]{1.00,0.00,0.00}{##1}}}
\expandafter\def\csname PY@tok@ge\endcsname{\let\PY@it=\textit}
\expandafter\def\csname PY@tok@gs\endcsname{\let\PY@bf=\textbf}
\expandafter\def\csname PY@tok@gp\endcsname{\let\PY@bf=\textbf\def\PY@tc##1{\textcolor[rgb]{0.00,0.00,0.50}{##1}}}
\expandafter\def\csname PY@tok@go\endcsname{\def\PY@tc##1{\textcolor[rgb]{0.53,0.53,0.53}{##1}}}
\expandafter\def\csname PY@tok@gt\endcsname{\def\PY@tc##1{\textcolor[rgb]{0.00,0.27,0.87}{##1}}}
\expandafter\def\csname PY@tok@err\endcsname{\def\PY@bc##1{\setlength{\fboxsep}{0pt}\fcolorbox[rgb]{1.00,0.00,0.00}{1,1,1}{\strut ##1}}}
\expandafter\def\csname PY@tok@kc\endcsname{\let\PY@bf=\textbf\def\PY@tc##1{\textcolor[rgb]{0.00,0.50,0.00}{##1}}}
\expandafter\def\csname PY@tok@kd\endcsname{\let\PY@bf=\textbf\def\PY@tc##1{\textcolor[rgb]{0.00,0.50,0.00}{##1}}}
\expandafter\def\csname PY@tok@kn\endcsname{\let\PY@bf=\textbf\def\PY@tc##1{\textcolor[rgb]{0.00,0.50,0.00}{##1}}}
\expandafter\def\csname PY@tok@kr\endcsname{\let\PY@bf=\textbf\def\PY@tc##1{\textcolor[rgb]{0.00,0.50,0.00}{##1}}}
\expandafter\def\csname PY@tok@bp\endcsname{\def\PY@tc##1{\textcolor[rgb]{0.00,0.50,0.00}{##1}}}
\expandafter\def\csname PY@tok@fm\endcsname{\def\PY@tc##1{\textcolor[rgb]{0.00,0.00,1.00}{##1}}}
\expandafter\def\csname PY@tok@vc\endcsname{\def\PY@tc##1{\textcolor[rgb]{0.10,0.09,0.49}{##1}}}
\expandafter\def\csname PY@tok@vg\endcsname{\def\PY@tc##1{\textcolor[rgb]{0.10,0.09,0.49}{##1}}}
\expandafter\def\csname PY@tok@vi\endcsname{\def\PY@tc##1{\textcolor[rgb]{0.10,0.09,0.49}{##1}}}
\expandafter\def\csname PY@tok@vm\endcsname{\def\PY@tc##1{\textcolor[rgb]{0.10,0.09,0.49}{##1}}}
\expandafter\def\csname PY@tok@sa\endcsname{\def\PY@tc##1{\textcolor[rgb]{0.73,0.13,0.13}{##1}}}
\expandafter\def\csname PY@tok@sb\endcsname{\def\PY@tc##1{\textcolor[rgb]{0.73,0.13,0.13}{##1}}}
\expandafter\def\csname PY@tok@sc\endcsname{\def\PY@tc##1{\textcolor[rgb]{0.73,0.13,0.13}{##1}}}
\expandafter\def\csname PY@tok@dl\endcsname{\def\PY@tc##1{\textcolor[rgb]{0.73,0.13,0.13}{##1}}}
\expandafter\def\csname PY@tok@s2\endcsname{\def\PY@tc##1{\textcolor[rgb]{0.73,0.13,0.13}{##1}}}
\expandafter\def\csname PY@tok@sh\endcsname{\def\PY@tc##1{\textcolor[rgb]{0.73,0.13,0.13}{##1}}}
\expandafter\def\csname PY@tok@s1\endcsname{\def\PY@tc##1{\textcolor[rgb]{0.73,0.13,0.13}{##1}}}
\expandafter\def\csname PY@tok@mb\endcsname{\def\PY@tc##1{\textcolor[rgb]{0.40,0.40,0.40}{##1}}}
\expandafter\def\csname PY@tok@mf\endcsname{\def\PY@tc##1{\textcolor[rgb]{0.40,0.40,0.40}{##1}}}
\expandafter\def\csname PY@tok@mh\endcsname{\def\PY@tc##1{\textcolor[rgb]{0.40,0.40,0.40}{##1}}}
\expandafter\def\csname PY@tok@mi\endcsname{\def\PY@tc##1{\textcolor[rgb]{0.40,0.40,0.40}{##1}}}
\expandafter\def\csname PY@tok@il\endcsname{\def\PY@tc##1{\textcolor[rgb]{0.40,0.40,0.40}{##1}}}
\expandafter\def\csname PY@tok@mo\endcsname{\def\PY@tc##1{\textcolor[rgb]{0.40,0.40,0.40}{##1}}}
\expandafter\def\csname PY@tok@ch\endcsname{\let\PY@it=\textit\def\PY@tc##1{\textcolor[rgb]{0.25,0.50,0.50}{##1}}}
\expandafter\def\csname PY@tok@cm\endcsname{\let\PY@it=\textit\def\PY@tc##1{\textcolor[rgb]{0.25,0.50,0.50}{##1}}}
\expandafter\def\csname PY@tok@cpf\endcsname{\let\PY@it=\textit\def\PY@tc##1{\textcolor[rgb]{0.25,0.50,0.50}{##1}}}
\expandafter\def\csname PY@tok@c1\endcsname{\let\PY@it=\textit\def\PY@tc##1{\textcolor[rgb]{0.25,0.50,0.50}{##1}}}
\expandafter\def\csname PY@tok@cs\endcsname{\let\PY@it=\textit\def\PY@tc##1{\textcolor[rgb]{0.25,0.50,0.50}{##1}}}

\def\PYZbs{\char`\\}
\def\PYZus{\char`\_}
\def\PYZob{\char`\{}
\def\PYZcb{\char`\}}
\def\PYZca{\char`\^}
\def\PYZam{\char`\&}
\def\PYZlt{\char`\<}
\def\PYZgt{\char`\>}
\def\PYZsh{\char`\#}
\def\PYZpc{\char`\%}
\def\PYZdl{\char`\$}
\def\PYZhy{\char`\-}
\def\PYZsq{\char`\'}
\def\PYZdq{\char`\"}
\def\PYZti{\char`\~}
% for compatibility with earlier versions
\def\PYZat{@}
\def\PYZlb{[}
\def\PYZrb{]}
\makeatother


    % Exact colors from NB
    \definecolor{incolor}{rgb}{0.0, 0.0, 0.5}
    \definecolor{outcolor}{rgb}{0.545, 0.0, 0.0}



    
    % Prevent overflowing lines due to hard-to-break entities
    \sloppy 
    % Setup hyperref package
    \hypersetup{
      breaklinks=true,  % so long urls are correctly broken across lines
      colorlinks=true,
      urlcolor=urlcolor,
      linkcolor=linkcolor,
      citecolor=citecolor,
      }
    % Slightly bigger margins than the latex defaults
    
    \geometry{verbose,tmargin=1in,bmargin=1in,lmargin=1in,rmargin=1in}
    
    

    \begin{document}
    
    
    \maketitle
    
    

    
    \hypertarget{linear-regression-with-multiple-variables}{%
\section{Linear Regression with Multiple
Variables}\label{linear-regression-with-multiple-variables}}

\hypertarget{multivariate-linear-regression}{%
\subsection{Multivariate Linear
Regression}\label{multivariate-linear-regression}}

\hypertarget{multiple-features}{%
\subsubsection{Multiple Features}\label{multiple-features}}

Linear regression with multiple variables is also known as
``\textbf{multivariate linear regression}''.

notation: \[
\begin{aligned}x_j^{(i)} &= \text{value of feature } j \text{ in the }i^{th}\text{ training example} \newline x^{(i)}& = \text{the input (features) of the }i^{th}\text{ training example} \newline m &= \text{the number of training examples} \newline n &= \text{the number of features} \end{aligned}
\]

hypothesis and \textbf{vectorization} \[
h_\theta (x) = \theta_0 + \theta_1 x_1 + \theta_2 x_2 + \theta_3 x_3 + \cdots + \theta_n x_n
\] \[
\begin{aligned}
h_\theta(x) & =\theta_0 + \theta_1 x_1 + \theta_2 x_2 + \theta_3 x_3 + \cdots + \theta_n x_n \\
& = \begin{bmatrix}\theta_0 &\theta_1 & \cdots & \theta_n\end{bmatrix}\begin{bmatrix}x_0 \newline x_1 \newline \vdots \newline x_n\end{bmatrix} \\
&= \theta^T x
\end{aligned}
\] Remark: Note that for convenience reasons in this course we assume
\(x^{(i)}_0=1\) for \((i∈1,…,m)\). This allows us to do matrix
operations with \(\theta\) and \(x\). Hence making the two vectors
`\(θ\)' and \(x^{(i)}\) match each other element-wise (that is, have the
same number of elements: \(n+1\)).{]}

\hypertarget{gradient-descent-for-multiple-variables}{%
\subsubsection{\texorpdfstring{\textbf{Gradient Descent for Multiple
Variables}}{Gradient Descent for Multiple Variables}}\label{gradient-descent-for-multiple-variables}}

The gradient descent equation itself is generally the same form; we just
have to repeat it for our \(n\) features:

\begin{longtable}[]{@{}c@{}}
\toprule
\begin{minipage}[b]{0.97\columnwidth}\centering
Gradient Descent for Multiple Variables\strut
\end{minipage}\tabularnewline
\midrule
\endhead
\begin{minipage}[t]{0.97\columnwidth}\centering
\(\begin{aligned}& \text{repeat until convergence:} \; \lbrace \newline \; & \theta_j := \theta_j - \alpha \frac{1}{m} \sum\limits_{i=1}^{m}(h_\theta(x^{(i)}) - y^{(i)}) \cdot x_j^{(i)} \; & \text{for j := 0...n}\newline &\rbrace\end{aligned}\)\strut
\end{minipage}\tabularnewline
\bottomrule
\end{longtable}

The following image compares gradient descent with one variable to
gradient descent with multiple variables:

\begin{figure}
\centering
\includegraphics{./assets/MYm8uqafEeaZoQ7hPZtKqg_c974c2e2953662e9578b38c7b04591ed_Screenshot-2016-11-09-09.07.04.png}
\caption{img}
\end{figure}

\hypertarget{gradient-descent-in-practice-i---feature-scaling}{%
\subsubsection{Gradient Descent in Practice I - Feature
Scaling}\label{gradient-descent-in-practice-i---feature-scaling}}

\textbf{We can speed up gradient descent by having each of our input
values in roughly the same range.} This is because \(\theta\) will
descend quickly on small ranges and slowly on large ranges, and so will
oscillate inefficiently down to the optimum when the variables are very
uneven.

The way to prevent this is to modify the ranges of our input variables
so that they are all roughly the same. Ideally \$−1 ≤ x\_\{(i)\} ≤ 1 \$
or \(−0.5 ≤ x_{(i)} ≤ 0.5\)

These aren't exact requirements; we are only trying to speed things up.
\textbf{The goal is to get all input variables into roughly one of these
ranges, give or take a few.}

Two techniques to help with this are \textbf{feature scaling} and
\textbf{mean normalization}.

\begin{itemize}
\item
  \textbf{Feature scaling}: dividing the input values by the range of
  the input variable, resulting \$−1 ≤ x\_i ≤ 1 \$.
\item
  \textbf{Mean normalization}: subtracting the average value for an
  input variable from the values for that input variable resulting
  \(E(x_i)=0\)
\item
  To implement both: \[
  x_i := \dfrac{x_i - \mu_i}{s_i}
  \] Where \(μ_i\) is the \textbf{average} of all the values for feature
  \((i)\) and \(s_i\) is the \textbf{range} of values (max - min), or
  \(s_i\) is the \textbf{standard deviation}.
\end{itemize}

\hypertarget{gradient-descent-in-practice-ii---learning-rate}{%
\subsubsection{Gradient Descent in Practice II - Learning
Rate}\label{gradient-descent-in-practice-ii---learning-rate}}

\hypertarget{debugging-gradient-descent}{%
\paragraph{Debugging gradient
descent}\label{debugging-gradient-descent}}

Make a plot with \emph{number of iterations} on the x-axis. Now plot the
cost function, \(J(θ)\) over the number of iterations of gradient
descent. If \(J(θ)\) ever increases, then you probably need to decrease
\(\alpha\)

\hypertarget{automatic-convergence-test}{%
\paragraph{Automatic convergence
test}\label{automatic-convergence-test}}

Declare convergence if \(J(θ)\) decreases by less than \(\varepsilon\)
in one iteration, where \(\varepsilon\) is some small value such as
\(10^{−3}\). However in practice it's difficult to choose this threshold
value.

\begin{figure}
\centering
\includegraphics{./assets/FEfS3aajEea3qApInhZCFg_6be025f7ad145eb0974b244a7f5b3f59_Screenshot-2016-11-09-09.35.59.png}
\caption{img}
\end{figure}

It has been proven that if learning rate \(\alpha\) is sufficiently
small, then \(J(θ)\) will decrease on every iteration.

\begin{figure}
\centering
\includegraphics{./assets/rC2jGKgvEeamBAoLccicqA_ec9e40a58588382f5b6df60637b69470_Screenshot-2016-11-11-08.55.21.png}
\caption{img}
\end{figure}

To summarize:

\begin{itemize}
\tightlist
\item
  If \(\alpha\) is too small: slow convergence.
\item
  If \(\alpha\) is too large: may not decrease on every iteration and
  thus may not converge.
\end{itemize}

\hypertarget{features-and-polynomial-regression}{%
\subsubsection{Features and Polynomial
Regression}\label{features-and-polynomial-regression}}

We can improve our features and the form of our hypothesis function in a
couple different ways.

\textbf{We can combine multiple features into one.} For example, we can
combine \(x_1\) and \(x_2\) into a new feature \(x_3\) by taking
\(x_1\cdot x_2\).

\hypertarget{polynomial-regression}{%
\paragraph{Polynomial Regression}\label{polynomial-regression}}

Our hypothesis function need not be linear (a straight line) if that
does not fit the data well.

We can \textbf{change the behavior or curve} of our hypothesis function
by making it a quadratic, cubic or square root function (or any other
form).

For example, if our hypothesis function is
\(h_\theta(x)=\theta_0+\theta_1 x_1\) then we can create additional
features based on \(x_1\), to get the quadratic function
\(h_\theta(x)=\theta_0+\theta_1x_1+\theta_2x^2_1\) or the cubic function
\(h_\theta(x)=\theta_0+\theta_1x_1+\theta_2x^2_1+\theta_3x^3_1\)

In the cubic version, we have created new features \(x_2\) and \(x_3\)
where \(x_2=x^2_1\) and \(x_3=x^3_1\).

To make it a square root function, we could do:
\$h\_\theta(x)=\theta\_0+\theta\_1x\_1+\theta\_2 \sqrt {x_1} \$

One important thing to keep in mind is, if you choose your features this
way then feature scaling becomes very important.

\begin{quote}
e.g.~if \(x_1\) has range 1 - 1000 then range of \(x^2_1\) becomes 1 -
1000000 and that of \(x^3_1\) becomes 1 - 1000000000
\end{quote}

\hypertarget{computing-parameters-analytically}{%
\subsection{Computing Parameters
Analytically}\label{computing-parameters-analytically}}

\hypertarget{normal-equation}{%
\subsubsection{Normal Equation}\label{normal-equation}}

A second way to minimize \(J(\theta)\) is \textbf{Normal Equation}
method. We will minimize \(J(\theta)\) by explicitly taking its
derivatives with respect to the \(\theta_j\)'s, and setting them to
zero. This allows us to find the optimum \(\theta\) without iteration.
\[
\theta=(X^TX)^{-1}X^Ty
\] \textgreater{} Example: \textgreater{} \textgreater{}
\includegraphics{./assets/dykma6dwEea3qApInhZCFg_333df5f11086fee19c4fb81bc34d5125_Screenshot-2016-11-10-10.06.16.png}

There is no need to do feature scaling with the normal equation.

\hypertarget{comparison-between-gradient-descent-and-normal-equation}{%
\paragraph{Comparison between Gradient Descent and Normal
Equation}\label{comparison-between-gradient-descent-and-normal-equation}}

\begin{longtable}[]{@{}ll@{}}
\toprule
Gradient Descent & Normal Equation\tabularnewline
\midrule
\endhead
Need to choose \(\alpha\) & No need to choose alpha\tabularnewline
Needs many iterations & No need to iterate\tabularnewline
\(O(kn^2)\) & \(O(n^3)\), need to calculate inverse of
\(X^TX\)\tabularnewline
Works well when \(n\) is large & Slow if \(n\) is very
large\tabularnewline
\bottomrule
\end{longtable}

In practice, when \(n\) exceeds \textbf{10,000} it might be a good time
to go from a normal solution to an iterative process.

\hypertarget{normal-equation-noninvertibility}{%
\subsubsection{Normal Equation
Noninvertibility}\label{normal-equation-noninvertibility}}

When implementing the normal equation in octave we want to use the
\texttt{pinv} function rather than \texttt{inv}. The \texttt{pinv}
function will give you a value of \(\theta\) even if \(X^TX\) is not
invertible.

If \(X^TX\) is \textbf{noninvertible,} the common causes might be having
:

\begin{itemize}
\tightlist
\item
  \textbf{Redundant features}, where two features are very closely
  related (i.e.~they are linearly dependent)
\item
  \textbf{Too many features} (e.g.~m ≤ n). In this case, delete some
  features or use ``regularization'' (to be explained in a later
  lesson).
\end{itemize}

Solutions to the above problems include deleting a feature that is
linearly dependent with another or deleting one or more features when
there are too many features.


    % Add a bibliography block to the postdoc
    
    
    
    \end{document}
